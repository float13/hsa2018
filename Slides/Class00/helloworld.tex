\documentclass[pdf]{beamer}
\mode<presentation>{
	\usetheme{CambridgeUS}
	%\usecolortheme{whale}
	\usecolortheme{dolphin}
}

\usepackage{amsmath}
\usepackage{listings}
%\usepackage{textcomp}
%\usepackage[dvipsnames]{xcolor}
%\setbeamercolor{frametitle}{fg=Brown,bg=Blue!20}

%% preamble
\title{Intro to Coding}
\subtitle{Fall 2018 - Class 0}
\author{Doug Brantner}
\begin{document}

\AtBeginSection[]
{
    \begin{frame}{Contents}
    \tableofcontents[currentsection]
    \end{frame}
}

%% title frame
\begin{frame}
\titlepage
\end{frame}

\begin{frame}{Contents}
\tableofcontents
\end{frame}


\section{Course Overview}

%\begin{frame}{Introduction}
%\begin{itemize}
%\item Computers can only do what you tell them to.
%\item Turing Complete: You can make anything that's possible to make.
%\end{itemize}
%\end{frame}

\begin{frame}{Ground Rules}
\begin{itemize}
\item No such thing as a stupid question!
\item Never copy/paste!
\item Type every character of your program.
\end{itemize}
\end{frame}


\begin{frame}{Useful Supplies}
\begin{itemize}
\item Notebook
\item Pen/Pencil
\item USB Thumb Drive to save work
\end{itemize}
\end{frame}

\begin{frame}{Saving Work}
At the moment, we have 3 options:
\begin{itemize}
\item Desktop folder
\item Student Shared folder
\item USB Thumb Drive
\end{itemize}
You can also email your files to yourself, but this is not recommended.\\
%\\
Saving in at least 2 locations is recommended for a \textbf{backup}.\\
\textbf{\textit{Always back up your work!!!}}
\end{frame}


\section{Quick Tour of Linux}
\begin{frame}[fragile]{Make a New Folder}
\begin{itemize}
\item Open the Terminal
\item Follow along with these commands:
\begin{lstlisting}[language=Bash]
>> pwd
>> man pwd      # manual page, very helpful!!!
>> cd ~/Desktop
>> mkdir ____________       # new folder name
>> cd ____________          # same name
>> echo "Hello Linux" > hello.txt
>> cat hello.txt
\end{lstlisting}
\item This folder is where you will store your projects
\end{itemize}
\end{frame}



\section{Processing Basics}
\begin{frame}{Processing}
\begin{itemize}
\item Based on Java
\item Runs on most computers (PC, Mac, Linux)
\item Makes graphics, animation, games, etc. super easy
\item It's free! www.processing.org
\item Tons of Example programs included
\item Sometimes helpful to search for ``proce55ing''
\end{itemize}
\end{frame}


\begin{frame}[fragile]{Hello World!}
\begin{lstlisting}[language=Java]
void setup() {
    println("Hello World!");
}
\end{lstlisting}
\begin{itemize}
\item \textbf{setup} is a Function, or in Java, a \textbf{Method}
\item \textbf{void} means that \textbf{setup} has no return value
\item \textbf{Curly Braces} mark the beginning and end of \textbf{setup}
\item \textbf{println} is a function called by \textbf{setup}
\item \textbf{"Hello World!"} is a \textbf{String} argument to \textbf{println}
\item A \textbf{semicolon} marks the end of each instruction
\end{itemize}
\end{frame}


\begin{frame}[fragile]{Basic Structure}
\begin{lstlisting}[language=Java]
void setup() {
    // Setup runs once at the beginning
}

void draw() {
    // Draw loops infinitely
}
\end{lstlisting}
\end{frame}

%% normal frame
\begin{frame}{Reserved Words}
Reserved Words have special meanings in a Programming Language.\\
\begin{itemize}
\item \textbf{void} is a Java reserved word
    \begin{itemize}
        \item So are \textbf{int}, \textbf{float}, \textbf{double}, \textbf{String}, \textbf{for}, \textbf{while}, \textbf{if} %\textellipsis
        %\item So are \textbf{+}, \textbf{-}, \textbf{*}, \textbf{/}, \textbf{=}
        \item So are $\mathbf{+}, \mathbf{-}, \mathbf{*}, \mathbf{/}, \mathbf{=}$
        \item Parenthesis (), Square Brackets [], and Curly Braces \{\} all have different special meanings.
    \end{itemize}

\item \textbf{setup} and \textbf{draw} are Processing reserved words
    \begin{itemize}
        \item So are \textbf{width}, \textbf{height}, \textbf{color}, \textbf{print} %\textbf{for}, \textbf{while}, and \textbf{if}
    \end{itemize}
\item And lots more! See the Documentation for a full list
\item Capitalization matters!
\end{itemize}
\end{frame}

\begin{frame}[fragile]{Variables}
\begin{itemize}
\item A variable is a named value. In math, $x = 10$ or $y = 98.6$\\
\item In programming, a variable must have a datatype.
\begin{lstlisting}[language=Java]
    int x = 10;
    float y = 98.6;
    char c = 'A';
    String s = "Hello!";
    boolean b = true;
\end{lstlisting}
\item Reserved words CANNOT be used as variable names.
\item Variable names should be descriptive. \textbf{radius} is more useful than \textbf{r}
\end{itemize}

\end{frame}

\begin{frame}{Datatypes}
\begin{itemize}
\item Numbers
    \begin{itemize}
        \item \textbf{int:} integer numbers (no decimals) $-1, 0, 1, 2, 3, \ldots$
        \item \textbf{float:} floating-point (decimal) numbers $3.14, 2.72, 90.7, \ldots$
        \item \textbf{double:} bigger double-precision floating-point numbers
        \item Numbers can be negative or positive.
    \end{itemize}
\item Text
    \begin{itemize}
        \item \textbf{char:} a single character, with single quotes 
        \item \textbf{String:} multiple characters, with double quotes
    \end{itemize}
\item Logical
    \begin{itemize}
        \item \textbf{boolean:} a binary variable: either \textbf{true} or \textbf{false}
    \end{itemize}
\end{itemize}
\end{frame}


\begin{frame}[fragile]{Comments}
\begin{itemize}
\item Comments are notes to yourself, to explain what the code is doing
\begin{lstlisting}[language=Java]
    // This is a comment.
    
    /*   This is a 
     *   multi-
     *   line
     *   comment
     */
\end{lstlisting}
\item Comments are ignored by the compiler
\item Comments are also useful for temporarily disabling parts of code
    \begin{itemize}
        \item This is useful for debugging
    \end{itemize}
\item Write lots of comments! You'll thank yourself later!
\end{itemize}
\end{frame}

\begin{frame}{Printing to the Console}
\begin{itemize}
\item The \textbf{print} and \textbf{println} methods print text to the console, at the bottom of the Processing window
    \begin{itemize}
    \item \textbf{print} always continues on the same line
    %\item \textbf{println} always ends with an invisible \textbf{newline \textquotesingle \textbackslash n \textquotesingle} character, so any following statement will start on a new line
    \item \textbf{println} always ends with an invisible \textbf{newline \textbackslash n} character, so any following statement will start on a new line
    %\item \textbf{println} always ends with an invisible \begin{verbatim}newline '\n'\end{verbatim} character, so any following statement will start on a new line
    \end{itemize}
\item These are \textit{very} useful for debugging
\end{itemize}
\end{frame}

\begin{frame}{Functions in Math}
In math, a function takes an input, does something to it, and returns a value.\\
First, we have to define our function:
$$f(x) = 2x$$
Then we can assign the return value to a new variable $y$
$$y = f(x)$$
% TODO numeric examples w/ "pause" in between
\end{frame}

\begin{frame}{Black Box Functions}
A function can also be thought of as a \textbf{black box}. This means that we know the inputs and the outputs, but we may not know exactly what happens on the inside.\\
For example, a washing machine takes an input of dirty clothes, washes them, and then returns clean clothes.
$$ \text{wetClothes} = \text{washer}(\text{dirtyClothes})$$
Likewise, a dryer takes in wet clothes, dries them, and returns dry clothes.
$$ \text{cleanClothes} = \text{dryer}(\text{wetClothes})$$
We know how to use the machines, even if we don't know how they work on the inside.
\end{frame}

% TODO delete this?
\begin{frame}{Functions II}
We can combine multiple functions in one line of code.
$$  y = f(x) $$
$$  z = g(y) $$
is equivalent to writing
$$  z = g(f(x))    $$
This allows us to eliminate an intermediate variable
$$ \text{cleanClothes} = \text{dryer}(\text{washer}(\text{dirtyClothes}))$$
But at the expense of being harder to read.
\end{frame}

\begin{frame}[fragile]{Java Functions: \textit{Methods}}
Functions are the basic building blocks of code. In Java, they are called \textbf{methods}, and in other languages they are called sub-routines.
\begin{lstlisting}[language=Java]
// this function takes no arguments 
// and returns no values
void setup() {
    size(640, 320);
}

// this function takes an integer argument,
// then returns its square
int square(int x) { 
    return x*x;     
}
\end{lstlisting}
\end{frame}

\begin{frame}[fragile]{Arguments and Return Values}
Many functions accept \textbf{input parameters} or \textbf{arguments}. These are variables that are passed to the function when it is called.
\begin{lstlisting}[language=Java]
void setup() {  // No input parameters
    int y = square(10);     // now y == 100
}

int square(int x) {   // 1 integer input
    return x*x;       // returns an integer
}
\end{lstlisting}
Many functions also \textbf{return} values. The return datatype is specified in the function definition.
\end{frame}


\begin{frame}[fragile]{Void Methods}
A method that does not return a value is called a \textbf{void} method.
\begin{lstlisting}[language=Java]
void setup() {
    size(320, 640);
}

void draw() {
    // lots of fun stuff happens here
}
\end{lstlisting}
Void methods are just as powerful as other functions; they just don't need to return anything when they're done.
\end{frame}


\begin{frame}[fragile]{Variable Scope}
\begin{itemize}
% TODO add a global var for example
\item The curly braces $\{\ldots\}$ mark the beginning and end of the method, or the \textbf{scope} of the method.
\begin{lstlisting}[language=Java]
int triple(int x) {
    int y = 3*x;
    return y;
}
\end{lstlisting}
%\item $x$ is an \textbf{argument} or \textbf{parameter} that is passed to the method.
\item $y$ is a \textbf{local} variable. It is defined within the \textbf{scope} of the method.
\item Both $x$ and $y$ are only visible within the \textbf{scope} of the method. They are not accessible anywhere else in the program.
\item When the method finishes, the value of $y$ is returned, and $y$ itself disappears.
\end{itemize}
\end{frame}

\begin{frame}[fragile]{Global Variables}
\begin{itemize}
\item Global variables are defined outside of any method. Therefore, \textbf{every} method can see them.
\item And every method can also \textbf{modify} them.
    \begin{itemize}
        \item This can be very useful, but also very dangerous
        \item Take care when deciding if a variable should be global or not
    \end{itemize}
\begin{lstlisting}[language=Java]
int p, q;

void setup() {
    p = 100;
    q = p;
    size(p, q);
}

void draw() {
    if (p == q) { ... do something fun ... }
}

\end{lstlisting}
\end{itemize}
\end{frame}

\begin{frame}[fragile]{For Loop}
A \textbf{for} loop executes a specific number of times.
\begin{lstlisting}[language=Java]
for (int i = 0; i < 10; i++) {
    // do something ten times...
}
\end{lstlisting}
\end{frame}

\begin{frame}[fragile]{While Loop}
A \textbf{while} loop can run for an unknown number of loops.\\
You are responsible to \textbf{create and update} the loop variable.
\begin{lstlisting}[language=Java]
int i = 0;
while (i < 10) {
    // do something here...
    i++;    // this is very important!!
}

\end{lstlisting}
An \textbf{infinte loop} happens when the test criteria is never true.\\
\textbf{This is a very common bug!}
\end{frame}

\section{End}
\begin{frame}{Homework!}
\begin{itemize}
\item Download Processing (www.processing.org)
    \begin{itemize}
    \item You may need to install some Java dependencies too
    \item Let me know if you need help
    \end{itemize}
\item Try to run some of the examples, look for interesting ones
\item Look through the Tutorials \& Videos
\item Start thinking about project ideas
\end{itemize}
\end{frame}

\end{document}

